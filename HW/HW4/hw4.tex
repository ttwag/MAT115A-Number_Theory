\documentclass{article} % Defines the document class, article is commonly used
\usepackage[shortlabels]{enumitem}
\usepackage{amsmath}    % Allows for more advanced math formatting
\usepackage{amssymb}    % Provides additional mathematical symbols
\usepackage{amsthm}     % \qed
\usepackage{graphicx}   % image
\usepackage{float}      % image placement
\usepackage{hyperref}
\hypersetup{
    colorlinks=true,       % false: boxed links; true: colored links
    linkcolor=black,       % color of internal links
}
\usepackage[margin=1.5in]{geometry}

\begin{document}

\title{MAT115A HW4}
\author{Tao Wang}
\date{\today}

\maketitle
\section*{(1) }
\begin{quote}
    Find the least nonnegative solution of each system of congruences below.
\end{quote}

\bigskip
\noindent
\textbf{Question (a)}:

\begin{quote}
    $x \equiv 1 \mod 2$

    $x \equiv 2 \mod 3$

    $x \equiv 3 \mod 5$

    $x \equiv 4 \mod 7$
\end{quote}

\bigskip
\noindent
\textbf{Answer}:
\bigskip
\begin{quote}

    Since $(m_1, m_2, m_3, m_4) = 1$, we can use the Chinese Remainder Theorem to find the solution.


    M = $m_1 \times m_2 \times m_3 \times m_4 = 2 \times 3 \times 5 \times 7 = 210$
    \bigskip

    \begin{tabular}{|c|c|c|c|c|}
        \hline
        $n$ & $a_n$ & $m_n$ & $M_n$ & $M_n^{-1}$ \\
        \hline
        1   & 1     & 2     & 105   & 1          \\
        \hline
        2   & 2     & 3     & 70    & 1          \\
        \hline
        3   & 3     & 5     & 42    & 3          \\
        \hline
        4   & 4     & 7     & 30    & 4          \\
        \hline
    \end{tabular}

    \bigskip

    $x = (1 \times 105 \times 1 + 2 \times 70 \times 1 + 3 \times 42 \times 3 + 4 \times 30 \times 4) \mod 210$

    $ \text{  }= 1103 \mod 210 = 53$

    $\boxed{x \equiv 53 \mod 210}$

\end{quote}

\bigskip
\noindent
\textbf{Question (b)}:
\begin{quote}
    $3x \equiv 2 \mod 4$

    $4x \equiv 1 \mod 5$

    $6x \equiv 3 \mod 9$
\end{quote}
\bigskip
\noindent
\textbf{Answer}:
\begin{quote}
    \noindent To remove the coefficient in front of x, we can multiply each congruences by the inverse of the coefficient.

    $3x * 3^{-1} \equiv 2 * 3^{-1} \mod 4$

    $\implies x \equiv 2 \mod 4$

    \bigskip
    $4x * 4^{-1} \equiv 1 * 4^{-1} \mod 5$

    $\implies x \equiv 4 \mod 5$

    \bigskip
    $6x \equiv 3 \mod 9$

    $\implies 2x \equiv 1 \mod 3$

    $\implies 2x * 2^{-1}  \equiv 1 * 2^{-1} \mod 3$

    $\implies x \equiv 2 \mod 3$

    \bigskip
    Now, we have

    $x \equiv 2 \mod 3$

    $x \equiv 2 \mod 4$

    $x \equiv 4 \mod 5$

    \bigskip
    Therefore, $(m_1, m_2, m_3) = 1$, and we can use the Chinese Remainder Theorem to find the solution

    $M = 3 * 4 * 5 = 60$

    \begin{tabular}{|c|c|c|c|c|}
        \hline
        $n$ & $a_n$ & $m_n$ & $M_n$ & $M_n^{-1}$ \\
        \hline
        1   & 2     & 3     & 20    & 2          \\
        \hline
        2   & 2     & 4     & 15    & 3          \\
        \hline
        4   & 4     & 5     & 12    & 3          \\
        \hline
    \end{tabular}

    $x = (2 \times 20 \times 2 + 2 \times 15 \times 3 + 4 \times 12 \times 3) \mod 60$

    $\text{  } = (80 + 90 + 144) \mod 60 = 14$

    $\boxed{x \equiv 14 \mod 60}$
\end{quote}

\bigskip

\bigskip
\noindent
\textbf{Question (c)}:
\begin{quote}
    $x \equiv 3 \mod 6$

    $x \equiv 7 \mod 10$

    $x \equiv 12 \mod 15$
\end{quote}

\pagebreak
\noindent
\textbf{Answer}:
\begin{quote}
    $x \equiv 3 \mod 6$

    $\implies x \equiv 1 \mod 2 \text{ and } x \equiv 0 \mod 3$

    $x \equiv 7 \mod 10$

    $\implies x \equiv 1 \mod 2 \text{ and } x \equiv 2 \mod 5$

    $x \equiv 12 \mod 15$

    $\implies x \equiv 0 \mod 3 \text{ and } x \equiv 2 \mod 5$
    \bigskip

    Therefore, we have

    $x \equiv 1 \mod 2$

    $x \equiv 0 \mod 3$

    $x \equiv 2 \mod 5$
    \bigskip

    Since $(2, 3, 5) = 1$, we can use the Chinese Remainder Theorem to find the solution.

    $M = 2 * 3 * 5 = 30$

    \begin{tabular}{|c|c|c|c|c|}
        \hline
        $n$ & $a_n$ & $m_n$ & $M_n$ & $M_n^{-1}$ \\
        \hline
        1   & 1     & 2     & 15    & 1          \\
        \hline
        2   & 0     & 3     & 10    & 1          \\
        \hline
        4   & 2     & 5     & 6     & 1          \\
        \hline
    \end{tabular}

    x = $(1 \times 15 \times 1 + 0 \times 10 \times 1 + 2 \times 6 \times 1) \mod 30$

    $\text{  } = 27$

    $\boxed{x \equiv 27 \mod 30}$
\end{quote}

\bigskip

\section*{(2)}
\begin{quote}
    Use Wilson's Theorem to find the least nonnegative residue modulo $m$ for each integer $n$ below.
\end{quote}

\bigskip
\noindent
\textbf{Question (a)}:
\begin{quote}
    $n = 30!, m = 31$
\end{quote}

\bigskip
\noindent
\textbf{Answer}:
\begin{quote}
    Choose $p = 31$. Then by Wilson's Theorem,

    $(31 - 1)! \equiv 30 \mod 31$

    $\boxed{30! \equiv 30 \mod 31}$

\end{quote}

\bigskip
\noindent
\textbf{Question (b)}:
\begin{quote}
    $n = 21!, m = 23$
\end{quote}

\bigskip
\noindent
\textbf{Answer}:
\begin{quote}
    Choose $p = 23$. Then by Wilson's Theorem,

    $(23 - 2)! \equiv 1 \mod 23$

    $\boxed{21! \equiv 1 \mod 23}$

\end{quote}

\bigskip
\noindent
\textbf{Question (c)}:
\begin{quote}
    $n = \frac{31!}{22!}, m = 11$
\end{quote}

\bigskip
\noindent
\textbf{Answer}:
\begin{quote}
    Notice that

    \[31 \equiv 9 \mod 11\]
    \[30 \equiv 8 \mod 11\]
    \[\vdots\]
    \[23 \equiv 1 \mod 11\]

    Therefore,

    \[\frac{31!}{22!} \equiv 9! \mod 11\]

    Choose $p = 11$. Then by Wilson's Theorem,

    $\boxed{\frac{31!}{22!} \equiv 9! \equiv (11 - 2)! \equiv 1 \mod 11}$
\end{quote}

\section*{(3)}
\begin{quote}
    Let $p$ be an odd prime number.
\end{quote}

\bigskip
\noindent
\textbf{Question (a)}:
\begin{quote}
    Prove that $\left(\left(\frac{p-1}{2}\right)!\right)^2 \equiv (-1)^{\frac{p + 1}{2}}$
\end{quote}

\pagebreak
\noindent
\textbf{Answer}:
\begin{quote}
    \[(p - 1)! = (p - 1) \times (p - 2) \times \dots \times \left(\frac{p + 1}{2}\right) \times \left(\frac{p - 1}{2}\right)!\]

    Notice that
    \[(p - 1) \equiv -1 \mod p\]
    \[(p - 2) \equiv -2 \mod p\]
    \[\vdots\]
    \[\left(\frac{p + 1}{2}\right) \equiv -\left(\frac{p - 1}{2}\right) \mod p\]

    This means that

    \[(p - 1) \times (p - 2) \times \dots \times \left(\frac{p + 1}{2}\right) \equiv (-1)^{\left(\frac{p - 1}{2}\right)} \times \left(\frac{p - 1}{2}\right)! \mod p\]

    If we multiply $\left(\frac{p -1}{2}\right)!$ to both sides of the expression above, we get

    \[(p - 1)! \equiv (-1)^{\left(\frac{p - 1}{2}\right)} \times \left(\left(\frac{p - 1}{2}\right)!\right)^2 \mod p\]

    Since $p$ is a prime number,

    \[(p - 1)! \equiv -1 \mod p\]

    by the Wilson's Theorem.

    Therefore,

    \[(-1)^{\left(\frac{p - 1}{2}\right)} \times \left(\left(\frac{p - 1}{2}\right)!\right)^2 \equiv -1 \mod p\]

    Multiply both sides by $(-1)^{\frac{p - 1}{2}}$, we get

    \[(-1)^{p - 1} \times \left(\left(\frac{p - 1}{2}\right)!\right)^2 \equiv (-1)^{\frac{p + 1}{2}} \mod p\]

    Since $p$ is odd, $p - 1$ is even, and $-1$ raised to an even power is 1.

    Therefore,

    \[\boxed{\left(\left(\frac{p - 1}{2}\right)!\right)^2 \equiv (-1)^{\frac{p + 1}{2}} \mod p}\]
\end{quote}

\pagebreak
\noindent
\textbf{Question (b)}:
\begin{quote}
    If $p \equiv 1 \mod 4$, prove that $\left(\frac{p-1}{2}\right)!$ is a solution of the quadratic congruence $x^2 \equiv -1 \mod p$.
\end{quote}

\bigskip
\noindent
\textbf{Answer}:
\begin{quote}
    We were given
    \[p - 1 = 4k\]
    for some $k \in \mathbb{Z}$

    Then,
    \[p = 4k + 1\]

    and

    \[\frac{p + 1}{2} = 2k + 1\]

    By the result from part (a),

    \[\left(\left(\frac{p - 1}{2}\right)!\right)^2 \equiv (-1)^{2k + 1} \equiv -1 \mod p\]

    Therefore, $\left(\frac{p-1}{2}\right)!$ is a solution of the quadratic congruence $x^2 \equiv -1 \mod p$.

\end{quote}

\bigskip
\noindent
\textbf{Question (c)}:
\begin{quote}
    If $p \equiv 3 \mod 4$, prove that $\left(\frac{p-1}{2}\right)!$ is a solution of the quadratic congruence $x^2 \equiv 1 \mod p$.
\end{quote}

\bigskip
\noindent
\textbf{Answer}:
\begin{quote}
    We were given
    \[p - 3 = 4k\]
    for some $k \in \mathbb{Z}$

    Then,
    \[p = 4k + 3\]

    and

    \[\frac{p + 1}{2} = 2k + 2\]

    By the result from part (a),

    \[\left(\left(\frac{p - 1}{2}\right)!\right)^2 \equiv (-1)^{2k + 2} \equiv 1 \mod p\]

    Therefore, $\left(\frac{p-1}{2}\right)!$ is a solution of the quadratic congruence $x^2 \equiv 1 \mod p$.
\end{quote}

\end{document}
