\documentclass{article} % Defines the document class, article is commonly used
\usepackage[shortlabels]{enumitem}
\usepackage{amsmath}    % Allows for more advanced math formatting
\usepackage{amssymb}    % Provides additional mathematical symbols
\usepackage{amsthm}     % \qed
\usepackage{graphicx}   % image
\usepackage{float}      % image placement
\usepackage{hyperref}
\hypersetup{
    colorlinks=true,       % false: boxed links; true: colored links
    linkcolor=black,       % color of internal links
}
\usepackage[margin=1.5in]{geometry}

\begin{document}

\title{MAT115A HW5}
\author{Tao Wang}
\date{\today}

\maketitle

\section*{(1)}
\begin{quote}
    Find the least nonnegative residue modulo m of each integer n below.
\end{quote}
\bigskip
\noindent
\textbf{(a)}:
\begin{quote}
    \[n = 29^{202}, m = 13\]
\end{quote}

\bigskip
\noindent
\textbf{Answer}:
\begin{quote}
    \[(29, 13) = 1\]
    \[\phi(13) = 12\]
    \[202 = (16)(12) + 10\]

    \begin{center}
        Since $(29, 13) = 1$, by Euler's Theorem,
    \end{center}
    \[29^{12} \equiv 1 (\text{mod } 13)\]
    \begin{center}
        Therefore,
    \end{center}
    \[29^{12} \equiv 1 \equiv (29^{12})^{16} (\text{mod } 13)\]

    \[29^{202} \equiv (29^{12})^{16} 29^{10} \equiv 29^{10} \equiv 3^{10} \equiv 3(\text{mod } 13)\]

    \[\boxed{29^{202} \equiv 3 (\text{mod } 13)}\]

\end{quote}
\bigskip
\noindent
\textbf{(b)}:
\begin{quote}
    \[n = 79^{79}, m = 9\]
\end{quote}

\bigskip
\noindent
\textbf{Answer}:
\begin{quote}
    \[(79, 9) = 1\]
    \[\phi(9) = 6\]
    \[79 = 6(13) + 1\]
    \begin{center}
        Since $(79, 9) = 1$, by Euler's Theorem, $79^{6} \equiv 1 (\text{mod } 9)$

        \[7^{79} \equiv 79^{6(13)}79 \equiv 79 \equiv 7 (\text{mod } 9)\]
        \[\boxed{79^{79} \equiv 7 (\text{mod } 9)}\]
    \end{center}

\end{quote}
\bigskip
\noindent
\textbf{(c)}:
\begin{quote}
    \[n = 99^{99999}, m = 26\]
\end{quote}

\bigskip
\noindent
\textbf{Answer}:
\begin{quote}
    \[(99, 26) = 1\]
    \[\phi(26) = 12\]
    \[99999 = (8333)(12) + 3\]
    Since $(99, 26) = 1$, by the Euler's Theorem, $99^{12} \equiv 1 (\text{mod }26)$

    \[(99^{12})^{8333} (99^3) \equiv 99^3 \equiv 21^3 \equiv (25)(21) \equiv 5 (\text{mod }26)\]
    \[\boxed{99^{99999} \equiv 5 (\text{mod }6)}\]
\end{quote}

\section*{(2)}

\bigskip
\noindent
\textbf{Proposition}:

\begin{quote}
    $645 = 3 \cdot 5 \cdot 43$ is a pseudoprime (to base 2).
\end{quote}

\bigskip
\noindent
\textbf{Proof}:
\begin{quote}
    645 is a composite because $645 = 3 \cdot 5 \cdot 43$.

    \[2^2 \equiv 1 (\text{mod }3)  \implies 2^{644} \equiv 1 (\text{mod }3)\]
    \[2^4 \equiv 1 (\text{mod }5)  \implies 2^{644} \equiv 1 (\text{mod }5)\]
    \[2^{42} \equiv 1 (\text{mod }43) \implies 2^{644} \equiv 1 (\text{mod }43)\]

    By the Chinese Remainder Theorem,

    \[2^{644} \equiv 1 (\text{mod }645)\]

    and

    \[\boxed{2^{645} \equiv 2 (\text{mod }645)}\]
\end{quote}
\qed
\bigskip

\section*{Exercise (3)}

\bigskip
\noindent
\textbf{Proposition}:
\begin{quote}
    $2821 = 7 \cdot 13 \cdot 31$ is a Carmichael number.
\end{quote}

\bigskip
\noindent
\textbf{Proof}:
\begin{quote}
    We'll show the proposition is true by Theorem 3.1.42.

    First, 2821 is composed of more than 2 distinct primes.

    Then,

    \[(7 - 1) \mid 2820\]
    \[(13 - 1) \mid 2820\]
    \[(31 - 1) \mid 2820\]

    As a result, 2821 is a carmichael number.
\end{quote}

\qed
\bigskip

\section*{Exercise (4)}

\bigskip
\noindent
\textbf{Proposition}:
\begin{quote}
    Let p and q be distinct odd prime numbers. Prove $p^{q-1} + q^{p-1} \equiv (\text{mod }pq)$.
\end{quote}

\bigskip
\noindent
\textbf{Proof}:
\begin{quote}
    By the Euler Theorem,
    \[p^{q-1} \equiv 1 (\text{mod }q)\]
    \[q^{p-1} \equiv 1 (\text{mod }p)\]

    By the definition of modulo,
    \[p^{q-1} \equiv 0 (\text{mod }p)\]
    \[q^{p-1} \equiv 0 (\text{mod }q)\]

    From the results above, for $m, n \in \mathbb{Z}$,

    \[p^{q-1} - 1 = qm\]
    \[q^{p - 1} = qn\]

    and

    \[p^{q-1} + q^{p-1} - 1 = q (m + n)\]
    \[\implies p^{q-1} + q^{p-1} \equiv 1 (\text{mod }q)\]

    Similarly,

    \[p^{q-1} + q^{p-1} \equiv 1 (\text{mod }p)\]

    Since p and q are co-prime, by the Chinese Remainder Theorem,

    \[p^{q-1} + q^{p-1} \equiv 1 (\text{mod }pq)\]

\end{quote}
\qed
\bigskip

\section*{Exercise (5)}

\bigskip
\noindent
\textbf{Proposition}:
\begin{quote}
    Let p be a prime number. Prove that $2^p - 1$ is either a prime or a pseudoprime (to the base 2).
\end{quote}

\bigskip
\noindent
\textbf{Proof}:
\begin{quote}
    Since $p \nmid 2$, by Fermat's Little Theorem, $2^{p - 1} \equiv 1 (\text{mod }p)$.
    \[2^{p-1} = 1 + mp \implies 2^p = 2 + 2 mp \implies 2^p - 2 = 2mp\]

    Therefore,

    \[2^{2^p - 2} = 2^{2mp} = (2^{p})^{2m}\]

    By the definition of modulo equivalence,

    \[2^p \equiv 1(\text{mod }2^p - 1)\]
    \[\implies (2^{p})^{2m} \equiv 1 (\text{mod }2^p - 1)\]
    \[\implies 2^{2^p - 2} \equiv 1 (\text{mod }2^p - 1)\]
    \[\implies 2^{2^p - 1} \equiv 2 (\text{mod }2^p - 1)\]

    As a result, when $2^p - 1$ is a composite, it must be a pseudoprime.
\end{quote}
\qed
\bigskip

\section*{Exercise (6)}

\bigskip
\noindent
\textbf{Proposition}:
\begin{quote}
    Let n be an integer not divisible by 3. Prove that $n^7 \equiv n (\text{mod }63)$.
\end{quote}

\bigskip
\noindent
\textbf{Proof}:
\begin{quote}
    Since $3 \nmid n$, $(n, 7) = 1$ and $(n, 9) = 1$.

    By Euler's Theorem,

    \[n^{\phi(7)} \equiv 1 (\text{mod }7)\]
    \[n^6 \equiv 1 (\text{mod }7)\]
    \[n^7 \equiv n (\text{mod }7)\]

    and

    \[n^{\phi(9)} \equiv 1 (\text{mod }9)\]
    \[n^6 \equiv 1 (\text{mod }9)\]
    \[n^7 \equiv n (\text{mod }9)\]

    Since (7, 9) = 1, by the Chinese Remainder Theorem,

    \[n^7 \equiv n (\text{mod }63)\]
\end{quote}

\qed
\bigskip

\end{document}

