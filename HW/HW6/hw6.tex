\documentclass{article} % Defines the document class, article is commonly used
\usepackage[shortlabels]{enumitem}
\usepackage{amsmath}    % Allows for more advanced math formatting
\usepackage{amssymb}    % Provides additional mathematical symbols
\usepackage{amsthm}     % \qed
\usepackage{graphicx}   % image
\usepackage{float}      % image placement
\usepackage{hyperref}
\hypersetup{
    colorlinks=true,       % false: boxed links; true: colored links
    linkcolor=black,       % color of internal links
}
\usepackage[margin=1.5in]{geometry}
\usepackage{siunitx}

\begin{document}

\title{MAT115A HW6}
\author{Tao Wang}
\date{\today}

\maketitle

\section*{(1)}

\bigskip
\noindent
\textbf{Question}:
\begin{quote}
    Find the order of 9 modulo 17.
\end{quote}

\bigskip
\noindent
\textbf{Answer}:
\begin{quote}
    \[\{9^1, 9^2, 9^3, 9^4, 9^5, 9^6, 9^7, 9^8\} (\text{mod }17) = \{9, 13, 15, 16, 8, 4, 2, 1\}\]
    By Definition 4.1.1, $\text{ord}_m a$ is the $n$ where $a^n \equiv 1 (\text{mod }m)$.
    Since $9^8 \equiv 1 (\text{mod } 17)$ and 8 is the least positive integer to satisfy this property, $\boxed{\text{ord}_{17} 9 = 8}$.
\end{quote}

\bigskip

\section*{(2)}

\bigskip
\noindent
\textbf{Question}:
\begin{quote}
    Find all incongruent primitive roots modulo 18.
\end{quote}

\bigskip
\noindent
\textbf{Answer}:
\begin{quote}
    \[\{5^1, 5^2, 5^3, 5^4, 5^5, 5^6\} (\text{mod } 18) = \{5, 7, 17, 13, 11, 1\}\]

    By Corollary 4.1.14.2, the number of incongruent primitive roots modulo 18 is $\phi(\phi(18)) = 2$.

    Also,

    \[\{11^1, 11^2, 11^3, 11^4, 11^5, 11^6\} (\text{mod } 18) = \{5, 7, 17, 13, 11, 1\}\]

    Therefore, the two incongruent primitive roots modulo 18 are $\boxed{\text{5 and 11}}$.

\end{quote}

\section*{(3)(a)}

\bigskip
\noindent
\textbf{Proposition}:
\begin{quote}
    Let $m$ be a positive integer and let $a, b$ be integers relatively prime to $m$ with $(\text{ord}_m a, \text{ord}_m b) = 1$. Prove that $\text{ord}_m (ab) = (\text{ord}_m a)(\text{ord}_m b)$.
\end{quote}

\bigskip
\noindent
\textbf{Proof}:
\begin{quote}
    We have $\text{ord}_m a = x$ and $\text{ord}_m b = y$.

    Therefore, $a^x \equiv 1 (\text{mod }m)$ and $b^y \equiv 1(\text{mod }m)$.

    \[(ab)^{xy} = (a^x)^y (b^y)^x \equiv 1^y 1^x \equiv 1 (\text{mod }m)\]

    By Proposition 4.1.1, $\text{ord}_m (ab) \mid xy$ and $\text{ord}_m (ab) \mid (\text{ord}_m a)(\text{ord}_m b)$

    Also, let $n = \text{ord}_m (ab)$. Then,
    \[((ab)^n)^y = (a^{ny})(b^y)^n = a^{ny}\equiv 1 (\text{mod }m)\]

    This implies $x \mid ny$, which implies $x \mid n$ because $(x, y) = 1$. Similarly, we could show that $y \mid n$.

    Since $(x, y) = 1$, $x \mid n$ and $y \mid n$ implies $xy \mid n$ or $(\text{ord}_m a)(\text{ord}_m b) \mid \text{ord}_m (ab)$

    Since we've proven divisibility in both direction, $\text{ord}_m (ab) = (\text{ord}_m a)(\text{ord}_m b)$
\end{quote}
\qed
\bigskip

\section*{(3)(b)}

\bigskip
\noindent
\textbf{Question}:
\begin{quote}
    Show that $(\text{ord}_m a, \text{ord}_m b) = 1$ cannot be eliminated from part (a).
\end{quote}
\bigskip
\noindent
\textbf{Answer}:
\begin{quote}
    We need $(\text{ord}_m a, \text{ord}_m b) = 1$ to show that $(\text{ord}_m a)(\text{ord}_m b) \mid \text{ord}_m (ab)$.
\end{quote}

\bigskip
\section*{(4)}

\bigskip
\noindent
\textbf{Proposition}:
\begin{quote}
    Show that $r$ is a primitive root modulo the odd prime $p$ if and only if $r$ is an integer with $(r, p) = 1$ such that
    \[r^{\frac{p-1}{q}} \not\equiv 1 (\text{mod }p)\]
    for all prime divisors $q$ of $p - 1$.
\end{quote}

\bigskip
\noindent
\textbf{Proof}:
\begin{quote}
    We'll first show that $r$ is a primitive root modulo $p$ implies $(r, p) = 1$ and $r^{\frac{p-1}{q}} \not\equiv 1 (\text{mod }p)$

    $r$ is a primitive root modulo $p$ implies that $(r, p) = 1$ and $r^{\phi(p)} \equiv 1 (\text{mod }p)$.

    Since $\phi(p) = p - 1$, we have $r^{p-1} \equiv 1(\text{mod }p)$.
    Assume that $r^{\frac{p-1}{q}} \equiv 1 (\text{mod }p)$, then there's a contradiction because $\frac{p-1}{q} < p-1$ and $r$ is a primitive root guarantees that p - 1 is the smallest integer n to make $r^{n} \equiv 1 (\text{mod }p)$.

    Therefore, $r^{\frac{p-1}{q}} \not\equiv 1 (\text{mod }p)$.

    \bigskip
    Next, we'll show that the converse is true.
    By the Euler's Theorem, $(r, p) = 1$ implies $a^{\phi(p)} \equiv 1 (\text{mod }p)$.

    By Proposition 4.1.3, $\text{ord}_m r \mid p - 1$.

    Assume $\text{ord}_m r < p - 1$ and $p - 1 = bq$ for some integer b and the prime divisor q, then $(\text{ord}_m r )(a) = \frac{p - 1}{q}$ for some integer a.

    By Definition 4.1.1, $r^{\text{ord}_m r} \equiv 1 (\text{mod }p)$, so $r^{(\text{ord}_m r)(a)} = (r^{\text{ord}_m r})^a \equiv r^{\frac{p-1}{q}}\equiv 1 (\text{mod }p)$.

    This contradicts our hypothesis that $r^{\frac{p-1}{q}} \not\equiv 1 (\text{mod }p)$. Therefore, $\text{ord}_m r = p - 1 = \phi(p)$ and r is a primitive root.
\end{quote}

\qed
\bigskip

\section*{(5)}

\bigskip
\noindent
\textbf{Proposition}:
\begin{quote}
    Show that if $r$ is a primitive root modulo the positive integer $m$, then $\bar{r}$, the inverse of $r$ modulo $m$, is also a primitive root modulo $m$.
\end{quote}

\bigskip
\noindent
\textbf{Proof}:
\begin{quote}
    Since $\bar{r}$ is the inverse of $r$,
    \[(r)(\bar{r}) \equiv 1 (\text{mod }m) \]
    \[\implies ((r)(\bar{r}))^{\phi(m)} \equiv 1 (\text{mod }m)\]

    However, $r$ is a primitive root modulo m implies $r^{\phi(m)} \equiv 1 (\text{mod }m)$.

    Both statements are true if and only if $\bar{r}^{\phi(m)} \equiv 1 (\text{mod }m)$.

    \bigskip
    $\phi(m)$ must also be the least root for $\bar{r}$.

    Assume that there exists $k < \bar{r}$, then $r^k \equiv 1 (\text{mod }m)$ holds because $(r)(\bar{r}) \equiv 1 (\text{mod }m)$. However, this contradicts with the fact the $r$ is a primitive root.

    \bigskip
    As a result, $\bar{r}$ is also a primitive root modulo $m$.




\end{quote}

\qed
\bigskip

\end{document}
