\documentclass{article} % Defines the document class, article is commonly used
\usepackage[shortlabels]{enumitem}
\usepackage{amsmath}    % Allows for more advanced math formatting
\usepackage{amssymb}    % Provides additional mathematical symbols
\usepackage{amsthm}     % \qed
\usepackage{graphicx}   % image
\usepackage{float}      % image placement
\usepackage{hyperref}
\hypersetup{
    colorlinks=true,       % false: boxed links; true: colored links
    linkcolor=black,       % color of internal links
}
\usepackage[margin=1.5in]{geometry}
\usepackage{siunitx}

\begin{document}

\title{MAT115A HW7}
\author{Tao Wang}
\date{\today}

\maketitle

\section*{Exercise (1)}

\bigskip
\noindent
\textbf{Question}:
\begin{quote}
    Let $n = 4861$. Use $4860 = 2^2 \cdot 3^5 \cdot  5$ and $x = 11$ to verify that $n$ is prime via
    Luca’s converse of Fermat’s Little Theorem or its corollary.
\end{quote}

\bigskip
\noindent
\textbf{Answer}:
\begin{quote}
    Since

    \[11^{4860} \equiv 1(\text{mod }4861)\]

    \[11^{\frac{4860}{2}}\not \equiv 1(\text{mod }4861)\]

    \[11^{\frac{4860}{3}}\not \equiv 1(\text{mod }4861)\]

    \[11^{\frac{4860}{5}}\not \equiv 1(\text{mod }4861)\]

    $4861$ is prime via Luca's converse of Fermat's Little Theorem.
\end{quote}

\section*{Exercise (2)}

\bigskip
\noindent
\textbf{Question}:
\begin{quote}
    If the two most common letters in a long ciphertext, encrypted by an affine
    transformation $C \equiv aP + b$ mod 26 are $W$ and $B$, respectively, then what are
    the most likely values for a and b?
\end{quote}

\bigskip
\noindent
\textbf{Answer}:
\begin{quote}
    Based on the given information, we can set up the following equations

    \[4a + b \equiv 22 (\text{mod }26)\]
    \[19a + b \equiv 1 (\text{mod }26)\]

    Plug $b \equiv 22 - 4a (\text{mod }26)$ into the second equation, we get $15a + 22 \equiv 1(\text{mod }26)$.
    Then, $15a \equiv -21 \equiv 5 (\text{mod }26)$. Since $15^{-1} \equiv 7 (\text{mod }26)$, $a \equiv 35 (\text{mod }26) = 9$

    Plug $a = 9$ back into the first equation, we get $36 + b \equiv 22(\text{mod }26)$. Therefore, $b \equiv -14 \equiv 12 (\text{mod }26)$.

    $\boxed{a = 9} \text{ and } \boxed{b = 12}$
\end{quote}

\section*{Exercise (3)}

\bigskip
\noindent
\textbf{Question}:
\begin{quote}
    What is the plaintext message that corresponds to the ciphertex
    \[13 \text{ } 11 \text{ } 11 \text{ } 02\]

    that is produced using modular exponentiation with modulus p = 29 and en-
    cryption exponent e = 17?
\end{quote}

\bigskip
\noindent
\textbf{Answer}:
\begin{quote}
    $p = 29$ and $e = 17$.

    We have $17d \equiv 1 (\text{mod }28)$ and $d \equiv 5 (\text{mod }28)$

    $P \equiv 13^5 \equiv 6(\text{mod }29)$ and $P \equiv 11^5 \equiv 14(\text{mod }29)$ and $P \equiv 2^5 \equiv 3(\text{mod }29)$

    Therefore, the plaintext message is 6 14 14 3, or $\boxed{G O O C}$
\end{quote}

\section*{Exercise (4)}

\bigskip
\noindent
\textbf{Question}:
\begin{quote}
    What is the ciphertext that is produced when RSA encryption with $N = 77$
    and $e = 7$ is used to encrypt the message “BEST”.
\end{quote}

\bigskip
\noindent
\textbf{Answer}:
\begin{quote}
    BEST = 01 04 18 19

    $N = 77$ and $e = 7$

    $C \equiv 1 ^ 7 \equiv 1 (\text{mod }77)$


    $C \equiv 4 ^ 7 \equiv 60 (\text{mod }77)$


    $C \equiv 18 ^ 7 \equiv 39 (\text{mod }77)$


    $C \equiv 19 ^ 7 \equiv 68(\text{mod }77)$

    Therefore, the ciphertext is $\boxed{01 \text{ }60\text{ } 39\text{ } 68}$.
\end{quote}


\section*{Exercise (5)}

\bigskip
\noindent
\textbf{Question}:
\begin{quote}
    What is the plaintext message that corresponds to the ciphertext
    \[01 \text{ } 49 \text{ } 49 \text{ } 10\]
    produced by the $RSA$ encryption with $N = 77$ and $e = 43$.
\end{quote}

\bigskip
\noindent
\textbf{Answer}:
\begin{quote}
    $N = 77$ and $e = 43$

    $ed \equiv 1 (\text{mod }\phi(77)) \implies 43d \equiv 1 (\text{mod }60)$

    $d \equiv 7 (\text{mod }60)$

    Therefore, $P \equiv 1^7 \equiv 1(\text{mod }77)$ and $P \equiv 49^7 \equiv 14(\text{mod }77)$ and $P \equiv 10^7 \equiv 10 (\text{mod }77)$

    The plaintext is 01 14 14 10 or $\boxed{B \text{ } O \text{ } O \text{ } K}$
\end{quote}
\end{document}
