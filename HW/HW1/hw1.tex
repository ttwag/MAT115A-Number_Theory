\documentclass{article} % Defines the document class, article is commonly used

\usepackage{amsmath}    % Allows for more advanced math formatting
\usepackage{amssymb}    % Provides additional mathematical symbols
\usepackage{amsthm}     % \qed
\usepackage{graphicx}   % image
\usepackage{float}      % image placement

\begin{document}

\title{MAT115A HW1}
\author{Tao Wang}
\date{\today}

\maketitle

\section*{Exercise 1(a)}

\bigskip
\noindent
\textbf{Proposition}: $0 \mid 10$

\bigskip
\noindent
\textbf{True or False?} False

\bigskip
\noindent
\textbf{Disproof}:
\begin{align*}
     & 0 \mid 10                                       &                                              \\
     & \iff (\exists m \in \mathbb{Z})(10 = 0 \cdot m) & \text{(Definition of Divisibility (2.1.1))}  \\
     & \iff (\exists m \in \mathbb{Z})(10 = 0)         & \text{(Proposition 1.0.1)}                   \\
     & \iff \text{False}                               & \text{(Contradicts Reflexivity of Equality)} \\
\end{align*}

Since $0 \mid 10 \iff \text{False}$, we know that $0 \mid 10$ must be false.
\qed
\bigskip

\section*{Exercise 1 (b)}

\bigskip
\noindent
\textbf{Proposition}: $14 \mid 2024$

\bigskip
\noindent
\textbf{True or False?} False

\bigskip
\noindent
\textbf{Disproof}:
\begin{align*}
     & 14 \mid 2024                                                    &                                                    \\
     & \iff (\exists m \in \mathbb{Z})(2024 = 14 \cdot m)              & \text{(Definition of Divisibility (2.1.1))}        \\
     & \iff (\exists m \in \mathbb{Z})\left(\frac{2024}{14} = m\right) & \text{(Division on both sides of equality)}        \\
     & \iff \text{False}                                               & \text{(m cannot be an integer and not an integer)} \\
\end{align*}

Since $14 \mid 2024 \iff \text{False}$, we know $14 \mid 2024$ must be False.
\qed
\bigskip

\section*{Exercise 1 (c)}

\bigskip
\noindent
\textbf{Proposition}: $17 \mid 998189$

\bigskip
\noindent
\textbf{True or False?} True.

\bigskip
\noindent
\textbf{Proof}:

\begin{align*}
     & 17 \mid 998189                                       &                                             \\
     & \iff (\exists m \in \mathbb{Z})(998189 = 17 \cdot m) & \text{(Definition of Divisibility (2.1.1))} \\
     & \iff \text{True}                                     & (\text{Choose } m \text{ to be } 58717)     \\
\end{align*}

\noindent Therefore, we've shown that $17 \mid 998189$ is equivalent to true.
\qed
\bigskip

\section*{Exercise 2 (a)}

\bigskip
\noindent
\textbf{Proposition}: $(a,\space b,\space c,\space \in \mathbb{Z}) \left((a \mid b) \land (c \mid d) \right) \implies a + c \mid b + d$.

\bigskip
\noindent
\textbf{True or False?} False.

\bigskip
\noindent
\textbf{Counter-example}: $a = 3, \space b = 6, \space c = 1, \space d = 3$

$3 \mid 6$ and $1 \mid 3$, but $4 \nmid 9$.
\qed
\bigskip

\section*{Exercise 2 (b)}

\bigskip
\noindent
\textbf{Proposition}: $(a,\space b,\space c,\space \in \mathbb{Z}) \left((a \mid b) \land (c \mid d) \right) \implies ac \mid bd$.

\bigskip
\noindent
\textbf{True or False?} True.

\bigskip
\noindent
\textbf{Proof}:
\begin{quote}
    Assume $(a,\space b,\space c,\space \in \mathbb{Z})$, $a \mid b$ and $c \mid d$. Then, for some $m, \space n \in \mathbb{Z}$, $b = a\cdot m$ and $d = c \cdot n$.
    If we multiply the two equations, we get,

    \[b \cdot d = a \cdot m \cdot c \cdot n\]

    Since multiplication is associative, we have,

    \[bd = ac (m n)\]

    This implies $ac \mid bd$.
\end{quote}

\qed
\bigskip

\section*{Exercise 2 (c)}

\bigskip
\noindent
\textbf{Proposition}: $(a,\space b,\space c,\space \in \mathbb{Z}) \left((a \nmid b) \land (b \nmid c) \right) \implies a \nmid c$

\bigskip
\noindent
\textbf{True or False?} False.

\bigskip
\noindent
\textbf{Counter-example}: $a = 3, \space b = 5, \space c = 6$

$3 \nmid 5$ and $5 \nmid 6$, but $3 \mid 6$.

\qed
\bigskip

\section*{Exercise 3}

\bigskip
\noindent
\textbf{Proposition}: $(\forall n)(5 \mid n^5 - n)$

\bigskip
\noindent
\textbf{True or False?} True

\bigskip
\noindent
\textbf{Proof}:

\begin{quote}
    We'll prove this proposition through the induction principle.
    For the base case, assume $n = 0$. We have $5 \mid 0$, which is true because we can choose the other divisor to be zero.
    Therefore, we can assume that $5 \mid k^5 - k$ as the inductive hypothesis.
    \bigskip

    We have two inductive cases, n = k + 1 for the positive integers and n = k - 1 for the negative integers, where $k \in \mathbb{Z}$.
    \bigskip

    For the $n = k + 1$ case, we have
    \[5 \mid (k + 1)^5 - (k + 1)\]
    If we expand the power, we get
    \[5 \mid k^5 + 5k^4 +10k^3 + 10k^2 + 4k\]
    We can add k and subtract k to the polynomial
    \[5 \mid k^5 - k + 5k^4 +10k^3 + 10k^2 + 5k\]
    \[= 5 \mid k^5 - k + 5(k^4 + 2k^3 + 2k^2+k)\]
    We see that 5 divides $5(k^4 + 2k^3 + 2k^2+k)$ because $(k^4 + 2k^3 + 2k^2+k)$ is an integer.
    Also, $5 \mid k^5 - k$ from the base case.

    Therefore, it follows from Proposition 2.1.5 that $5 \mid k^5 - k + 5(k^4 + 2k^3 + 2k^2+k)$ is true.

    \bigskip
    For the $n = k - 1$ case, we can follow the same procedure and get

    \[5 \mid k^5 - k + 5(k -k^4 + 2k^3 - 2k^2)\]

    Proposition 2.1.5 again shows that $5 \mid k^5 - k + 5(k -k^4 + 2k^3 - 2k^2)$ is true.

    As a result, we've shown that $(\forall n)(5 \mid n^5 - n)$ is true in the base case and the two inductive cases that cover the positive and negative integers. Therefore, it is true for all $n$.
\end{quote}
\qed
\bigskip

\section*{Exercise 4 (a)}

\bigskip
\noindent
\textbf{Proposition}: 201

\bigskip
\noindent
\textbf{Prime or Not Prime?} Not Prime

$201 = 3 (67)$.
\bigskip

\section*{Exercise 4 (b)}

\bigskip
\noindent
\textbf{Proposition}: 211

\bigskip
\noindent
\textbf{Prime or Not Prime?} Prime
\bigskip

\section*{Exercise 4 (c)}

\bigskip
\noindent
\textbf{Proposition}: 213

\bigskip
\noindent
\textbf{Prime or Not Prime?} Not Prime

$213 = 3 (71)$.
\bigskip

\section*{Exercise 4 (d)}

\bigskip
\noindent
\textbf{Proposition}: 221

\bigskip
\noindent
\textbf{Prime or Not Prime?} Not Prime

$221 = 17 (13)$.
\bigskip

\section*{Exercise 5}

\bigskip
\noindent
\textbf{Proposition}: if a prime is in the arithmetic progression $3n + 1, n = 1, 2, 3, \dots$
then it is also in the arithmetic progression $6k + 1, k = 1, 2, 3, \dots$

\bigskip
\noindent
\textbf{Proof}:
\begin{quote}

    Assume a prime, $p$, is in $3n + 1$, where $n \in \mathbb{Z}^+$. Then, $p = 3n + 1$.

    Since $p$ is a prime number, we know that n must not be an odd positive integer. That's because If $(\exists m \in \mathbb{Z}_{\geq 0})(n = 2m + 1)$, then we have $p = 6m + 3 + 1$, which means $p$ is divisible by $2$. However, $p$ is a prime that cannot equal 2, so it should not be divisible by $2$ by the definition of prime numbers, so there's a contradiction.

    Therefore, $n$ must be an even positive integer. If $(\exists k \in \mathbb{Z}^+)(n = 2k)$, then $p = 3(2k) + 1 = 6k + 1$. We conclude that $(\exists k \in \mathbb{Z}^+)(p = 6k + 1)$, so $p$ is in the arithmetic progression, $6k + 1$, for $k = 1, 2, 3, \dots$
\end{quote}
\qed
\bigskip

\section*{Exercise 6(a)}

\bigskip
\noindent
\textbf{Proposition}: If $p$ is a prime, then $2^p - 1$ is a prime.

\bigskip
\noindent
\textbf{True or False?} False.

\bigskip
\noindent
\textbf{Counter-example}:$p = 11$.

$p = 11$ is prime, but $2^11 - 1 = 2047$, which is not a prime number because $2047 = 89(23)$.

\qed
\bigskip

\pagebreak
\section*{Exercise 6(b)}

\bigskip
\noindent
\textbf{Proposition}: If $2^p - 1$ is a prime, then $p$ is a prime.

\bigskip
\noindent
\textbf{True or False?} True.

\bigskip
\noindent
\textbf{Proof}:
\begin{quote}
    We will show the proposition is true by showing that its contrapositive is true.
    Therefore, if $p$ is not a prime, then $2^p - 1$ is not a prime.

    \bigskip
    We will assume that $p$ is a composite, so $p = a \cdot b$, where $1 < a, b < p$. Now we have $2^p - 1 = 2^{ab} - 1$, where $1 < a, b < p$.

    Recall that the sum of a finite geometric series is
    \[
        1 + r^2 + r^3 + \dots + r^{n - 1} = \frac{r^n - 1}{r - 1}, \quad \text{where } r \ne 1
    \]

    If $r = 2^a$ and $n = b$, we have

    \[
        1 + 2^{2a} + 2^{3a} + \dots + 2^{a(b - 1)} = \frac{2^{ab} - 1}{2^a - 1}, \quad \text{where } 2^a \ne 1
    \]

    Multiply $2^a - 1$ to both sides, we get
    \[
        2^{ab} - 1 = (2^a - 1)(1 + 2^{2a} + 2^{3a} + \dots + 2^{a(b - 1)}), \quad \text{where } 2^a \ne 1
    \]

    The divisor, $a$, is greater than 1, so $2^a \ne 1$. Therefore, we have

    \[2^{ab} - 1 = (2^a - 1)(1 + 2^{2a} + 2^{3a} + \dots + 2^{a(b - 1)}), \quad \text{where } 1 < a, b < p \]

    We've shown that $2^{ab}$ consists of two divisors that are greater than 1. Therefore, $2^{ab} - 1$ is not a prime.

    This is equivalent showing that if $2^p - 1$ is prime, then $p$ is prime.
\end{quote}
\qed
\bigskip

\end{document}
